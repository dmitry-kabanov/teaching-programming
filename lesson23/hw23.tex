\documentclass[a4paper,12pt]{article}
\usepackage[T2A]{fontenc}
\usepackage[utf8]{inputenc}
\usepackage[english,russian]{babel}
\usepackage[margin=2cm]{geometry}
\usepackage{fancyvrb}
\usepackage{inconsolata}
\usepackage{graphicx}

\newcommand\userinput[1]{\textbf{#1}}

\newcounter{problemnumber}

\newenvironment{problem}[1]
	{\addtocounter{problemnumber}{1}\arabic{problemnumber}. (#1)}
	{\vspace{6pt}}

\title{Домашняя работа 23}
\date{}
\begin{document}
\maketitle{}

\emph{Во всех программах размер экрана $640 \times 480$. Эти числа должны
    задаваться в виде констант в начале программы.}

\begin{problem}{Культин, 196}
    Написать программу, которая вычерчивает на экране кораблик, как показано
    на рисунке~\ref{fig:ship}.
\end{problem}

\begin{problem}{Культин, 197}
    Написать программу, которая вычерчивает на экране ракету, как показано на
    рисунке~\ref{fig:rocket}.
\end{problem}

\begin{problem}{Культин, 199}
    Написать программу, которая вычерчивает на экране узор из 100 окружностей
    случайного диаметра и цвета.
\end{problem}

\begin{problem}{Культин, 200}
    Написать программу, которая вычерчивает на экране узор ломаную линию,
    состоящую из 200 звеньев, окрашеных в разные цвета, выбираемые случайным
    образом, причём координаты звеньев тоже выбираются случайно.
\end{problem}

\begin{figure}
\centering
\begin{minipage}{.5\textwidth}
  \centering
  \includegraphics{ship.pdf}
  \caption{Иллюстрация к задаче 1.}
  \label{fig:ship}
\end{minipage}\hfill
\begin{minipage}{.5\textwidth}
  \centering
  \includegraphics{rocket.pdf}
  \caption{Иллюстрация к задаче 2.}
  \label{fig:rocket}
\end{minipage}
\end{figure}

\end{document}
