\documentclass[a4paper,12pt]{article}
\usepackage[T2A]{fontenc}
\usepackage[utf8]{inputenc}
\usepackage[english,russian]{babel}
\usepackage[margin=2cm]{geometry}
\usepackage{fancyvrb}
\usepackage{inconsolata}
\usepackage{graphicx}

\newcommand\userinput[1]{\textbf{#1}}

\newcounter{problemnumber}

\newenvironment{problem}[1]
	{\addtocounter{problemnumber}{1}\arabic{problemnumber}. (#1)}
	{\vspace{6pt}}

\title{Домашняя работа 27}
\date{}
\begin{document}
\maketitle{}

\emph{Во всех программах требуется написание собственных функций. Функции
    используются для разбиения задачи на подзадачи.}

\begin{problem}{Семакин, 6.1.А.1}
    Треугольник задан координатами своих вершин. Вычислить его площадь.
    Внутренняя организация программа такова: программа должна принимать от
    пользователя координаты вершин и выводить на экран площадь. Программа
    должна содержать две функции: одна вычисляет площадь, а вторая вычисляет
    стороны по координатам.
\end{problem}

\begin{problem}{Семакин, 6.1.А.2}
    Найти наибольший общий делитель и наименьшее общее кратное двух
    натуральных чисел, если дана формула

    \[
        НОК(A, B) = \frac{A \cdot B}{НОД(A, B)}.
    \]
\end{problem}

\begin{problem}{Семакин, 6.1.А.5}
    Найти сумму большего и меньшего из трёх чисел.
\end{problem}

\end{document}
