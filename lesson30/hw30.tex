\documentclass[a4paper,12pt]{article}
\usepackage[T2A]{fontenc}
\usepackage[utf8]{inputenc}
\usepackage[english,russian]{babel}
\usepackage[margin=2cm]{geometry}
\usepackage{fancyvrb}
\usepackage{inconsolata}
\usepackage{graphicx}

\newcommand\userinput[1]{\textbf{#1}}

\newcounter{problemnumber}

\newenvironment{problem}[1]
	{\addtocounter{problemnumber}{1}\arabic{problemnumber}. (#1)}
	{\vspace{6pt}}

\title{Домашняя работа 30}
\date{}
\begin{document}
\maketitle{}

\emph{Во всех программах рекомендую выделять логику обработки данных в
    функции. Можешь использовать глобальные массивы.}

\begin{problem}{Культин166}
    Написать программу, которая определяет количество учеников в классе, чей
    рост превышает средний. Рекомендуемый вид экрана во время работы программы
    приведён ниже (введённые пользователем данные выделены полужирным
    начертанием).

    \selectlanguage{english}
    \begin{Verbatim}[commandchars=\\\{\}]
        Program analyses the heights of pupils. To finish enter 0.
        -> \userinput{175}
        -> \userinput{170}
        -> \userinput{180}
        -> \userinput{168}
        -> \userinput{170}
        -> \userinput{0}

        The average height is 172.6 cm.
        2 pupils are taller than average height.
    \end{Verbatim}
    \selectlanguage{russian}
\end{problem}

\begin{problem}{Семакин7.A.6}
    Дана последовательность чисел $a_1, a_2, \dots, a_n$. Указать диапазон,
    в~котором находятся эти числа.
\end{problem}

\begin{problem}{Семакин7.A.7}
    Дана последовательность целых чисел $a_1, a_2, \dots, a_n$. Заменить все
    члены, которые больше заданного числа $Z$, этим числом. Вывести
    получившийся массив и количество сделанных замен.
\end{problem}

\begin{problem}{Семакин7.A.9}
    Дан массив действительных чисел. Определить, сколько в нём отрицательных,
    положительных и нулевых элементов.
\end{problem}

\begin{problem}{Семакин7.A.10}
    Дан массив целых чисел. Поменять местами наибольшее и наименьшее из этих
    чисел и вывести получившийся массив на экран.
\end{problem}

\begin{problem}{Семакин7.A.13}
    В заданном массиве целых чисел поменять местами соседние элементы, стоящие
    на чётных местах, с элементами, стоящими на нечётных местах.
\end{problem}

\end{document}
