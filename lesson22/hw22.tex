\documentclass[a4paper,12pt]{article}
\usepackage[T2A]{fontenc}
\usepackage[utf8]{inputenc}
\usepackage[english,russian]{babel}
\usepackage[margin=2cm]{geometry}
\usepackage{fancyvrb}
\usepackage{inconsolata}
\usepackage{graphicx}

\newcommand\userinput[1]{\textbf{#1}}

\newcounter{problemnumber}

\newenvironment{problem}[1]
	{\addtocounter{problemnumber}{1}\arabic{problemnumber}. (#1)}
	{\vspace{6pt}}

\title{Домашняя работа 22}
\date{}
\begin{document}
\maketitle{}

\emph{Во всех программах размер экрана $640 \times 480$. Эти числа должны
    задаваться в виде констант в начале программы.}

\begin{problem}{ASoJ, p. 130, 14}
    Доработать программу, которая выводит шахматную доску, так, чтобы она также
    рисовала начальное положение шашек. Шашки нарисовать кругами - чёрными
    с~заливкой и без заливки.
\end{problem}

\begin{problem}{ASoJ, p. 55, 7}
    Написать программу, которая рисует символ Олимпиады.
\end{problem}

\begin{problem}{ASoJ, p. 130, 11}
    Написать программу, которая рисует пирамиду из кирпичей, как показано
    на~рисунке~\ref{fig:p3}. Пирамида состоит из горизонтальных рядов
    кирпичей, количество кирпичей уменьшается на единицу в каждом ряде. Размер
    кирпича должен задаваться константами в начале программы:
    \texttt{BRICK\_WIDTH}, \texttt{BRICK\_HEIGHT}.
\end{problem}

\begin{figure}
\noindent\centering{
\includegraphics{p3.pdf}
}
\caption{Иллюстрация к задаче 3.}
\label{fig:p3}
\end{figure}

\end{document}
