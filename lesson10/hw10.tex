\documentclass[12pt,russian,draft]{article}
\usepackage[utf8x]{inputenc}
\usepackage[T1,T2A]{fontenc}
\usepackage[margin=2cm]{geometry}
\usepackage[final]{listings}
\usepackage{fancyvrb}
\usepackage[russian,english]{babel}

\title{Домашняя работа 10 <<Цикл for ещё раз>>}
\date{}
\begin{document}
\maketitle{}

\emph{Не забудь создать папку hw10 в папке dev!}

1. (Культин109) Написать программу, которая вводит с клавиатуры 5 дробных чисел
и вычисляет их среднее арифметическое.  Рекомендуемый вид экрана во время
работы програм\-мы приведён ниже. Данные, введённые пользователем.

\begin{Verbatim}
Enter 1 number: 5.4
Enter 2 number: 7.8
Enter 3 number: 3.0
Enter 4 number: 1.5
Enter 5 number: 2.3

The average is 4.00.
\end{Verbatim}

В качестве дополнительного упражнения доработай программу так, чтобы после
каждого номера выводился соответствующий суффикс, например, 1st вместо 1, 2nd
вместо 2 и т. п.

2. (Культин116) Написать программу, которая выводит таблицу значений функции
$y=|x|$. Диапазон изменения аргумента --- от $-4$ до $4$, шаг приращения
аргумента --- $0.5$.

3. (Культин119) Написать программу, которая выводит таблицу умножения, например
на 7 (число вводит пользователь, от 1 до 9): 

\begin{Verbatim}
7 x 1 =  1
7 x 2 = 14
7 x 3 = 21
...
7 x 8 = 56
7 x 9 = 63
\end{Verbatim}

4. (Культин125) Написать программу, которая выводит на экран изображение шахматной доски. Чёрные клетки отображать "звёздочкой", белые --- пробелом. Рекомендуемый вид экрана:
\begin{Verbatim}
* * * * 
 * * * *
* * * * 
 * * * *
* * * * 
 * * * *
* * * * 
 * * * *
\end{Verbatim}

\end{document}
