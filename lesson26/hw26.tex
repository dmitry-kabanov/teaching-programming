\documentclass[a4paper,12pt]{article}
\usepackage[T2A]{fontenc}
\usepackage[utf8]{inputenc}
\usepackage[english,russian]{babel}
\usepackage[margin=2cm]{geometry}
\usepackage{fancyvrb}
\usepackage{inconsolata}
\usepackage{graphicx}

\newcommand\userinput[1]{\textbf{#1}}

\newcounter{problemnumber}

\newenvironment{problem}[1]
	{\addtocounter{problemnumber}{1}\arabic{problemnumber}. (#1)}
	{\vspace{6pt}}

\title{Домашняя работа 26}
\date{}
\begin{document}
\maketitle{}

\emph{Во всех программах требуется написать функцию. Помимо функции решение
    должно содержать основную программу, которая вызывает эту функцию
    и~демонстрирует правильность работы функции.}

\begin{problem}{Культин, 176}
    Написать функцию, которая вычисляет объём цилиндра. Параметрами функции
    должны быть радиус и высота цилиндра.
\end{problem}

\begin{problem}{Культин, 177}
    Написать функцию, которая возвращает  максимальное из двух целых чисел,
    полученных в качестве аргумента.
\end{problem}

\begin{problem}{Культин, 178}
    Написать функцию, которая сравнивает два целых числа и возвращает результат
    сравнения в следующем виде: если первое число меньше второго, тогда
    результат $ -1 $; если первое число равно второму, тогда результат 0; если
    же первое число больше второго, тогда результат 1. Основная программа
    должна принимать числа от пользователя, и выводить текстовые сообщения на
    основе полученного из функции результата ($ -1 $, 0 или 1).
\end{problem}

\begin{problem}{моя}
    Написать программу, которая принимает от пользователя два целых
    положительных числа и возводит первое число в степень, показателем которой
    является второе число. Возведение в степень должно быть выполнено в виде
    функции.
\end{problem}

\end{document}
