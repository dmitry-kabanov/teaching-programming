\documentclass[12pt,russian,draft]{article}
\usepackage[utf8x]{inputenc}
\usepackage[T1,T2A]{fontenc}
\usepackage[margin=2cm]{geometry}
\usepackage[final]{listings}
\usepackage{fancyvrb}

\title{Домашняя работа 8}
\date{}
\begin{document}
\maketitle{}

\emph{Не забудь создать папку hw8 в папке dev!}

1. (Культин77) Написать программу вычисления площади кольца. Программа должна
проверять правильность исходных данных. Ниже представлен рекомендуемый
интерфейс программы:
\begin{Verbatim}
Enter ring radius: 3
Enter ring hole radius: 7

Error! Ring radius must be larger than the hole radius.
\end{Verbatim}

2. (Культин82) Написать программу проверку знания даты основания
Санкт-Петербурга. В~случае неверного ответа пользователя програама должна
выдавать правильный ответ. (\emph{Дату основания посмотри в~Википедии}.)

3. Пользователь вводит четырёхзначное число $abcd$, где каждая буква означает
соответ\-ствующую цифру. Нужно определить, являются ли числа $bcda$ и $bdac$
чётными. 

\end{document}
