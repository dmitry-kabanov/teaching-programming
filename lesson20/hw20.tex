\documentclass[a4paper,12pt,draft]{article}
\usepackage[T2A]{fontenc}
\usepackage[utf8]{inputenc}
\usepackage[english,russian]{babel}
\usepackage[margin=2cm]{geometry}
\usepackage{fancyvrb}
\usepackage{inconsolata}

\newcommand\userinput[1]{\textbf{#1}}

\newcounter{problemnumber}

\newenvironment{problem}[1]
	{\addtocounter{problemnumber}{1}\arabic{problemnumber}. (#1)}
	{\vspace{6pt}}

\title{Домашняя работа 20}
\date{}
\begin{document}
\maketitle{}

\emph{Во всех программах размер экрана $640 \times 480$. Эти числа должны
    задаваться в виде констант в начале программы.}

\begin{problem}{моя}
    Написать программу, которая выводит на экран 10 линий со случайными
    координатами. Линии должны иметь случайный цвет из палитры в 16 цветов.
    (Подсказка: для того чтобы задавать цвет, необязательно использовать
    цветовую константу, можно задавать цвет с помощью целого числа от 0 до 15.)
\end{problem}

\begin{problem}{моя}
    Вывести на экран изображение шахматной доски $8\times8$. Изображение
    должно быть в центре экрана. Размер клетки должны задаваться константой в
    начале программы.
\end{problem}

\begin{problem}{моя}
    Вывести на экран изображение для игры в крестики-нолики. Изображение
    должно быть в центре экрана. Размер клетки должны задаваться константой в
    начале программы.
\end{problem}

\end{document}
