\documentclass[12pt,russian,draft]{article}
\usepackage[utf8x]{inputenc}
\usepackage[T1,T2A]{fontenc}
\usepackage[margin=2cm]{geometry}
\usepackage[final]{listings}
\usepackage{fancyvrb}

\title{Домашняя работа 7}
\date{}
\begin{document}
\maketitle{}

\emph{Не забудь создать папку hw7 в папке dev! В задачах 1--3 программа должна
    печатать True на экране, если указанные высказывания является истинными,
    и~False~--- в противном случае.}

1. (Семакин1.3.14) Даны координаты левой верхней и~правой нижней вершин
пря\-мо\-уголь\-ника: $(x_1, y_1)$ и~$(x_2, y_2)$. Пользователь вводит координаты
точки $A(x, y)$. Нужно определить, принадлежит ли точка $A$ прямоугольнику.

2. (Семакин1.3.15) Цифры заданного четырёхзначного числа образуют строго
воз\-раста\-ющую последовательность.

3. (Семакин1.3.21) Цифра $M$ входит в~десятичную запись четырёхзначного
числа~N.

4. (Семакин2.1.2) Даны две точки $A(x_1, y_1)$ и~$B(x_2, y_2)$. Составить
алгоритм, опре\-деля\-ю\-щий, какая из этих точек находится ближе к~началу
координат.

5. (Культин79м) Написать программу решения квадратного уравнения:
\[
    ax^2 + bx + c = 0.
\]
Подразумевается, что $a \neq 0$.

6. (Культин80) Написать программу вычисления стоимости покупки с~учётом скидки.
Скидка в~10\% предоставляется, если сумма покупки больше 1000~руб. Программа
должна информировать пользователя о~том, что ему была предоставлена скидка.

\end{document}
