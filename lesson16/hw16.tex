\documentclass[a4paper,12pt,draft]{article}
\usepackage[T2A]{fontenc}
\usepackage[utf8]{inputenc}
\usepackage[english,russian]{babel}
\usepackage[margin=2cm]{geometry}
\usepackage{fancyvrb}
\usepackage{inconsolata}

\newcommand\userinput[1]{\textbf{#1}}

\newcounter{problemnumber}

\newenvironment{problem}[1]
	{\addtocounter{problemnumber}{1}\arabic{problemnumber}. (#1)}
	{\vspace{6pt}}

\title{Домашняя работа 16}
\date{}
\begin{document}
\maketitle{}

\begin{problem}{Культин134}
    Написать программу, которая <<задумывает>> число в диапазоне от~1 до~10
    включительно и~предлагает пользователю угадать число за 5~попыток.
    Рекомендуемый вид экрана во время работы программы:

    \selectlanguage{english}
    \begin{Verbatim}[commandchars=\\\{\}]
    Guess Number game.
    Computer "picked up" a secret number from 1 to 10 inclusive.
    You must guess this number. Max number of trials is 5.
    1. Enter your guess: \userinput{5}
    Too low.
    2. Enter your guess: \userinput{9}
    Too high.
    3. Enter your guess: \userinput{6}
    You win!
    \end{Verbatim}
    \selectlanguage{russian}
    
\end{problem}

\begin{problem}{Культин136}
    Написать программу, которая вычисляет $\pi$ с заданной пользователем
    точностью. Для этого воспользуйтесь тем, что значение частичной суммы ряда
    $1 - 1/3 + 1/5 - 1/7 + 1/9 - \dots$ при суммировании достаточно большого
    количества членов приближается к $\pi / 4$. Рекомендуемый вид экрана
    (значения, введённые пользователем выделены полужирным шрифтом):

    \selectlanguage{english}
    \begin{Verbatim}[commandchars=\\\{\}]
        Enter precision of PI computation: \userinput{0.001}
        The value of PI with precision 0.001000 is equal to 3.141589
        502 terms of series was summed.
    \end{Verbatim}
    \selectlanguage{russian}

\end{problem}

\begin{problem}{Семакин5.B.5}
    Натуральное число называется совершенным, если оно равно сумме всех своих
    делителей, включая единицу, но исключая само это число. Напечатать все
    совершенные числа, которые меньше заданного числа $N$.
\end{problem}

\begin{problem}{Семакин5.B.11}
    Найти наименьшее натуральное число $n$, которое можно представить в виде
    суммы кубов двух натуральных чисел.
\end{problem}

\begin{problem}{Семакин5.B.20}
    Последовательность Хэмминга образуют натуральные числа, не имеющие других
    простых делителей кроме 2, 3 и 5. Найти первые $N$ членов этой
    последовательности и сумму этих членов.
\end{problem}

\begin{problem}{Семакин5.B.27}
    Дано натуральное число $N$. Получить новое число $M$ посредством замены
    последней цифры числа $N$ наибольшей цифрой в~его записи. Например, если
    дано $N = 128452$, то $M = 128458$.
\end{problem}

\end{document}
