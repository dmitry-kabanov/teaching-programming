\documentclass[a4paper,12pt,draft]{article}
\usepackage[T2A]{fontenc}
\usepackage[utf8]{inputenc}
\usepackage[english,russian]{babel}
\usepackage[margin=2cm]{geometry}
\usepackage{fancyvrb}
\usepackage{inconsolata}

\newcommand\userinput[1]{\textbf{#1}}

\newcounter{problemnumber}

\newenvironment{problem}[1]
	{\addtocounter{problemnumber}{1}\arabic{problemnumber}. (#1)}
	{\vspace{6pt}}

\title{Домашняя работа 16}
\date{}
\begin{document}
\maketitle{}

\begin{problem}{ASoJ,p. 181}
    Написать игру в Craps. Игра начинается с того, что бросается два кубика и
    подсчитывается результат. Дальнейшая игра основывается на этом результате:

    \begin{itemize}
        \item{Результат 2, 3 или 12. Такой результат называется craps, и
                означает поражение.}
        \item{Результат 7 или 11. Такой результат называется natural, и
                означает выигрыш.}
        \item{Любой другой результат называется point. При получении point'а
                кубики продолжают бросаться, пока не произойдёт одно из двух:
                либо результатом очередного бросания будет тот же самый  point,
                что означает выигрыш игрока, либо результатом будет 7, что
                означает проигрыш игрока.}
    \end{itemize}
\end{problem}

\begin{problem}{SEE.CS106A, A2.5}
    Написать программу, которая считывает последовательность целых чисел с
    клавиатуры, а затем подсчитывает наименьший и наибольший элементы.
    Количество чисел в последовательности неопределено заранее. Для того,
    чтобы пользователь мог закончить ввод, нужно использовать специальное
    значение. Такое значение в программировании называется sentinel. Для
    данной задачи в качестве sentinel'а подходит значение 0.
    Примерный вариант (значения, вводимые пользователем, выделены полужирным
    шрифтом):

    \selectlanguage{english}
    \begin{Verbatim}[commandchars=\\\{\}]
    This program finds the largest and smallest numbers.
    To end the sequence, use value 0.
    ? \userinput{76}
    ? \userinput{23}
    ? \userinput{9}
    ? \userinput{-5}
    ? \userinput{2}
    ? \userinput{0}
    smallest: -5
    largest: 76
    \end{Verbatim}
    \selectlanguage{russian}

    Программа должна отрабатывать специальные случаи:

    \begin{itemize}
        \item{Если пользователь вводит всего лишь одно значение перед
                sentinel'ом, то программа должна вывести в качестве
                наибольшего и наименьшего элементов именно это значение.}
        \item{Если пользователь сразу же вводит sentinel, значит
                последовательность чисел пустая, и программа должна выводить
                соответствующее сообщение.}
    \end{itemize}
    
\end{problem}

\begin{problem}{ASoJ, p. 168}
    Написать программу, которая для двух целых чисел находит их наибольший
    общий делитель алгоритмом Евклида, основанном на делении. Суть алгоритма
    такова:

    \begin{itemize}
        \item{Разделить $x$ на $y$ и вычислить остаток. Остаток обозначим $r$.}
        \item{Если $r$ равен нулю, значит, ответ $y$.}
        \item{Если же $r$ не равен нулю, то присвоить $x$ значение $y$,
                присвоить $y$ значение $r$ и повторить всю процедуру.}
    \end{itemize}

    Также необходимо подсчитать число итераций цикла, которое требуется для
    вычисления НОД.
\end{problem}

\end{document}
