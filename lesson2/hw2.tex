\documentclass[12pt,russian,draft]{article}
\usepackage[utf8x]{inputenc}
\usepackage[T1,T2A]{fontenc}
\usepackage[margin=2cm]{geometry}
\usepackage{listings}

\title{Домашняя работа 2}
\date{}
\begin{document}
\maketitle{}

\emph{Для каждого задания в работе необходимо написать программу. Каждая
    программа должна выводить на экран результаты с поясняющими словами на
    английском языке, например: The area of the disc with radius r = 5 equals
    78.5.}

\emph{Для вычисления квадратного корня использовать функцию \texttt{Sqrt(X)},
    которая возвращает значение, равное квадратному корню из аргумента.}

\emph{Для вычисления квадрата числа можно использовать функцию \texttt{Sqr(X)},
    которая возвращает квадрат аргумента. Аргумент может быть выражением целого
    или вещественного типа.}

1.\ Найти площадь треугольника со сторонами $ a = 3 $, $ b = 7 $ и углом между
ними $ \gamma = 60^{\circ} $.

2.\ Найти площадь равнобедренной трапеции с основаниями $ a = 5 $, $ b = 3 $ и
углом $ \alpha = 40^{\circ} $ при большем основании.

3.\ Даны координаты трёх вершин треугольника $ () $, $ () $, $ () $. Найти его
периметр и площадь. Если непонятно, как решать, прочитай про формулу Герона.

4.\ Табличный вывод функции (форматированный вывод).

5.\ Вывести на экран первые четыре степени числа $ \pi $:
\begin{lstlisting}
PI: <some value>
PI^2: <some value 2>
PI^3: <some value 3>
PI^4: <some value 4>
\end{lstlisting}
\end{document}
