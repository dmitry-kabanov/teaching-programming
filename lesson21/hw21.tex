\documentclass[a4paper,12pt]{article}
\usepackage[T2A]{fontenc}
\usepackage[utf8]{inputenc}
\usepackage[english,russian]{babel}
\usepackage[margin=2cm]{geometry}
\usepackage{fancyvrb}
\usepackage{inconsolata}
\usepackage{graphicx}

\newcommand\userinput[1]{\textbf{#1}}

\newcounter{problemnumber}

\newenvironment{problem}[1]
	{\addtocounter{problemnumber}{1}\arabic{problemnumber}. (#1)}
	{\vspace{6pt}}

\title{Домашняя работа 21}
\date{}
\begin{document}
\maketitle{}

\emph{Во всех программах размер экрана $640 \times 480$. Эти числа должны
    задаваться в виде констант в начале программы.}

\begin{problem}{ASoJ, p.54, 4}
    Написать программу, которая рисует домик, как показано
    на~рисунке~\ref{fig:p1}.
\end{problem}

\begin{figure}
\noindent\centering{
\includegraphics{p1.pdf}
}
\caption{Иллюстрация к задаче 1}
\label{fig:p1}
\end{figure}

\begin{problem}{ASoJ, p. 54, 5}
    Написать программу, которая рисует лицо робота, как показано
    на~рисунке~\ref{fig:p2}.
    Глаза должны быть оранжевыми, нос чёрным, рот белым. Лицо робота должно
    быть светло-серого цвета.
\end{problem}

\begin{figure}
\noindent\centering{
\includegraphics{p2.pdf}
}
\caption{Иллюстрация к задаче 2}
\label{fig:p2}
\end{figure}

\begin{problem}{ASoJ, p. 54, 6}
    Написать программу, которая рисует изображение мишени, как показано
    на~рисунке~\ref{fig:p3}. Внешний и внутренний круги должны быть красного
    цвета.
\end{problem}

\begin{figure}
\noindent\centering{
\includegraphics{p3.pdf}
}
\caption{Иллюстрация к задаче 3}
\label{fig:p3}
\end{figure}

\end{document}
