\documentclass[a4paper,12pt,draft]{article}
\usepackage[T2A]{fontenc}
\usepackage[utf8]{inputenc}
\usepackage[english,russian]{babel}
\usepackage[margin=2cm]{geometry}
\usepackage{fancyvrb}
\usepackage{inconsolata}

\newcommand\userinput[1]{\textbf{#1}}

\newcounter{problemnumber}

\newenvironment{problem}[1]
	{\addtocounter{problemnumber}{1}\arabic{problemnumber}. (#1)}
	{\vspace{6pt}}

\title{Домашняя работа 19}
\date{}
\begin{document}
\maketitle{}

\begin{problem}{моя}
    Написать программу, которая считывает последовательность целых чисел
    с~клавиатуры, а затем подсчитывает сумму тех из этих чисел, которые
    являются простыми.  Количество чисел в последовательности не определено
    заранее. Для того, чтобы пользователь мог закончить ввод, нужно
    использовать sentinel.
\end{problem}

\begin{problem}{Семакин5.C.14}
    Дано натуральное число $n$. Если это не палиндром, реверсировать его цифры
    и сложить исходное число с числом, полученным в результате реверсирования.
    Если полученная при этом сумма не палиндром, то повторять указанные
    действия до тех пор, пока не получится палиндром. Например, для исходного
    числа 78 запишем: $78 + 87 = 165$; $165 + 561 = 726$; $726 + 627 = 1353$;
    $1353 + 3531 = 4884$. Число 4884 является палиндромом.
\end{problem}

\end{document}
