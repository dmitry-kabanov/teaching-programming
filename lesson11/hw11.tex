\documentclass[12pt,a4paper,draft]{article}
\usepackage[utf8x]{inputenc}
\usepackage[T1,T2A]{fontenc}
\usepackage[english,russian]{babel}
\usepackage[margin=2cm]{geometry}
\usepackage{fancyvrb}
\usepackage{inconsolata}

\newcommand\userinput[1]{\textbf{#1}}

\title{Домашняя работа 11 <<Цикл for ещё раз>>}
\date{}
\begin{document}
\maketitle{}

\emph{Не забудь создать папку hw11 в папке dev!}

1. (Культин127) Написать программу проверки знания таблицы умножения. Программа
должна вывести 5 примеров и выставить оценку: за 5 правильных ответов оценка
<<отлично>>, за 4 правильных ответа~--- <<хорошо>>, за 3~---
<<удовлетворительно>>, иначе <<неудовлетворительно>>. Примеры должны проверять
знание таблицы умножения от 1 до 9. Примеры должны генерироваться с
использованием функции \foreignlanguage{english}{\texttt{Random}}. Интерфейс
должен выглядеть так (полужирным выделены данные, которые вводит пользователь):

\selectlanguage{english}
\begin{Verbatim}[commandchars=\\\{\}]
5 x 3 = \userinput{15}
7 x 7 = \userinput{49}
7 x 8 = \textbf{52}
Error! 7 x 8 = 56
4 x 8 = \userinput{32}
9 x 3 = \userinput{27}

You grade is 'good'.
\end{Verbatim}
\selectlanguage{russian}

2. (Семакин4.1.2) Начав тренировки, спортсмен в первый день пробежал 10 км.
Каждый день он увеличивал дневную норму на 10 \% от нормы предыдущего дня.
Определить, какой суммарный путь пробежит спортсмен за $n$ дней. Число $n$
вводится с клавиатуры. Программа должна выводить суммарный путь за каждый
день, от первого до $n$-ого.

3. (Семакин4.1.3) Одноклеточная амёба каждые три часа делится на две клетки.
Определить, сколько амёб будет через 3, 6, 9, 12, \dots, $3n$ часов. Число $n$
вводится с клавиатуры.

4. (Семакин4.1.5) У гусей и кроликов вместе 64 лапы. Сколько может быть
кроликов и сколько гусей (указать все возможные сочетания).

5. (Семакин4.1.17) Найти сумму всех $n$-значных чисел ($1 \leq n \leq 4$).

6. (Семакин4.1.24) Вычислить количество точек с целочисленными координатами,
находящихся в круге с радиусом $R \geq 0 $. Радиус является вещественным
числом, которое пользователь вводит с клавиатуры.

\end{document}
