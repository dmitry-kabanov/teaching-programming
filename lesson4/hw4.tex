\documentclass[12pt,russian,draft]{article}
\usepackage[utf8x]{inputenc}
\usepackage[T1,T2A]{fontenc}
\usepackage[margin=2cm]{geometry}
\usepackage[final]{listings}
\usepackage{fancyvrb}

\title{Домашняя работа 4}
\date{}
\begin{document}
\maketitle{}

\emph{Для каждого задания в работе необходимо написать программу. Каждая
    программа должна выводить на экран результаты с поясняющими словами на
    английском языке, например: The area of the disc with radius r = 5 equals
    78.5.}

1. Выполнить следующую программу:
\begin{lstlisting}
    program error;
    uses crt;
    var
        i1: Integer;
        i2: Integer;
        i3: Integer;
    begin
        i1 := 30000;
        i2 := 30000;
        i3 := i1 + i2;

        WriteLn(i1, ' + ', i2, ' = ', i3);
        ReadKey;
    end.
\end{lstlisting}

Подумать, почему программа даёт, на первый взгляд, ошибочный ответ. Исправить
программу так, чтобы она давала правильный ответ.

2. (Культин56). Написать программу вычисления стоимости покупки, состоящей из
тет\-ра\-дей и~ка\-рандашей. Программа должна работать следующим образом:

\begin{verbatim}
Calculating the cost of buying exercise books and pencils.
Enter the number of exercise books: <пользователь вводит количество тетрадей>
Enter the price of an exercise book: <пользователь вводит цену тетради>
Enter the number of pencils: <пользователь вводит количество карандашей>
Enter the price of a pencil: <пользователь вводит цену карандаша>
\end{verbatim}

Затем программа вычисляет результат и~выводит его.

3. (Культин59) Написать программу вычисления площади треугольника, если
известна длина основания и~высота. Ниже представлен рекомендуемый интерфейс
взаимодействия с~пользователем:

\begin{verbatim}
Calculating the area of a triangle.
Enter the base (cm): <пользователь вводит длину основания>
Enter the height (cm): <пользователь вводит высоту>

The area is <result> sq. cm.
\end{verbatim}

4. (Культин60) Написать программу вычисления площади треугольника, если
известны длины двух его сторон и~величина угла между этими сторонами. Ниже
представлен реко\-мен\-ду\-емый вид экрана во время работы программы (данные,
введенные пользователем, выделены курсивом).

\begin{Verbatim}[commandchars=\|\{\}]
Calculating the area of a triangle.
Enter first side (cm): |emph{25}
Enter second side (cm): |emph{17}
Enter angle between sides: |emph{30}

The area of the triangle: 106.25 sq. cm.
\end{Verbatim}

5. (Культин72) Написать программу пересчёта величины временного интервала,
задан\-ного в минутах, в~величину, выраженную в~часах и~минутах. Ниже представлен
ре\-ко\-мен\-ду\-емый вид экрана во время работы программы (данные,
введенные пользователем, выделены курсивом).

\begin{Verbatim}[commandchars=\|\{\}]
Enter time interval in minutes: |emph{150}

150 minutes are 2 h 30 min
\end{Verbatim}

6. Написать программу, которая спрашивает у пользователя его имя и~фамилию (по
отдельности), а затем выводит приветствие \texttt{'Hello <firstName>
    <lastName>!'}.

7. Написать программу, которая спрашивает у пользователя целое число $x$,
а~затем выводит на экран результат вычисления $x+1$, $x+20$, $x-1$, $x-10$. Для
вычислений использовать операторы инкремента и декремента. (Подсказка:
понадобится дополнительная переменная, которая будет хранить исходное значение,
введённое пользователем.)

\end{document}
