\documentclass[a4paper,12pt,draft]{article}
\usepackage[T2A]{fontenc}
\usepackage[utf8]{inputenc}
\usepackage[english,russian]{babel}
\usepackage[margin=2cm]{geometry}
\usepackage{fancyvrb}
\usepackage{inconsolata}

\newcommand\userinput[1]{\textbf{#1}}

\newcounter{problemnumber}

\newenvironment{problem}[1]
	{\addtocounter{problemnumber}{1}\arabic{problemnumber}. (#1)}
	{\vspace{6pt}}

\title{Домашняя работа 18}
\date{}
\begin{document}
\maketitle{}

\begin{problem}{Семакин5.С.7}
    Для записи римских цифр используют символы I, V, X, L, C, D, M,
    обозначающие соответственно числа 1, 5, 10, 50, 100, 500, 1000. При этом
    один и тот же символ не может повторяться подряд больше трёх раз.
    Например, число 3 записывается как III, число 8 записывается как VIII, но
    для записи числа 4 используется запись IV, то есть когда левый символ
    обозначают меньшую цифру, то он вычитается из последующей цифры: V - I = 5
    - 1 = 4. Ещё один пример: число 29 будет записано римскими цифрами как
    XXIX. Написать программу, которая переведёт любое заданное число $1 \leq n
    \leq 3999$ в запись римскими цифрами. Не пытайся решить задачу всю сразу
    целиком. Начни с того, что напиши программу для перевода чисел от 1 до 10
    в запись римскими цифрами. Для того, чтобы лучше разобраться с римской
    записью, прочитай статью <<Римские числа>> в Википедии. Там есть таблица с
    примерами перевода.
\end{problem}

\begin{problem}{Семакин5.С.11}
    Найти целые числа, при возведении в квадрат которых получаются полиндромы.
    Например, $ 26^2 = 676 $.
\end{problem}

\end{document}
