\documentclass[12pt,russian]{article}
\usepackage[utf8x]{inputenc}
\usepackage[T1,T2A]{fontenc}
\usepackage[russian]{babel}
\usepackage[margin=2cm]{geometry}
\usepackage{listings}

\begin{document}
\section{Устройство компьютера}
Компьютер представляет собой аппаратно-программный комплекс для решения
формализованных задач. АПК означает, что компоненты компьютера разделяются на
оборудование --- внутренние устройства, внешние устройства,--- и программы.

Программа --- последовательность инструкций, которую компьютер может выполнить,
для получения результата, важного для человека.

\begin{itemize}
    \item asdfjkl;
    \item asdfjkl;
    \item asqwq.
\end{itemize}
Внутренние устройства:
\vspace{-\topsep}
\begin{itemize}
\setlength{\parskip}{0pt}
\setlength{\itemsep}{0pt plus 1pt}
\item{процессор — устройство, выполняющее все программы на
компьютере;}
\item{оперативная память (Random-Access Memory) — устройство, в котором
хранятся программы и данные, с которыми эти программы работают, тогда, когда
компьютер включён;}
\item{шина данных — устройство, которое позволяет передавать
данные между процессором и оперативной памятью.}
\end{itemize}

Процессор выполняет маленькую
часть инструкций, из которых состоит программа, загружая их из оперативной
памяти через шину, а также необходимые данные.

Внешние устройства:
\vspace{-\topsep}
\begin{itemize}
\setlength{\parskip}{0pt}
\setlength{\itemsep}{0pt plus 1pt}
\item{устройства ввода данных — мышь, клавиатура;}
\item{устройства отображения информации — монитор (дисплей), принтер;}
\item{устройства для долгосрочного хранения информации — жёсткий диск, накопители Flash;}
\item{другие устройства, такие как звуковая карта.}
\end{itemize}

\section{Введение в Turbo Pascal 7.0}
Turbo Pascal~--- это язык
программирования, специально созданный для обучения программированию. Он
поддерживает работу со многими типами данных и позволяет вводить свои
собственные.

Для того, чтобы разрабатывать программы было удобно, была создана среда
программирования Turbo Pascal. Эта среда позволяет писать код программы
и~запускать их на выполнение.

\section{Hello World}

\section{Структура программы}
\begin{lstlisting}
program <program name>;
uses <modules>;
const <constants>;
var <vars>;
begin
    <program body>
end.
\end{lstlisting}

\section{Целочисленный тип данных}

\section{Константы}

\end{document}
