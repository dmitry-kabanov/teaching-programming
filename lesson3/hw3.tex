\documentclass[12pt,russian,draft]{article}
\usepackage[utf8x]{inputenc}
\usepackage[T1,T2A]{fontenc}
\usepackage[margin=2cm]{geometry}
\usepackage[final]{listings}
\usepackage{fancyvrb}

\title{Домашняя работа 3}
\date{}
\begin{document}
\maketitle{}

\emph{Для каждого задания в работе необходимо написать программу. Каждая
    программа должна выводить на экран результаты с поясняющими словами на
    английском языке, например: The area of the disc with radius r = 5 equals
    78.5.}

1.\ Вычислить площадь треугольника со сторонами $ a = 3 $, $ b = 8.5 $ и~углом
между ними $ \gamma = 75^{\circ} $.

2.\ Вычислить площадь равнобедренной трапеции с основаниями $ a = 5 $, $ b = 3
$ и~углом $ \alpha = 40^{\circ} $ при большем основании. Результат представить
с 6 знаками после запятой.

3.\ Вывести на экран значения функции $ y = 1 / x^2 $ при значениях $ x \in 
\{1, 2, 3\} $. Вывод должен быть таким:
\begin{Verbatim}[fontfamily=tt]
 x     y   
===========
 1   1.000 
 2   0.250 
 3   0.111 
\end{Verbatim}
то есть значения $ y $ должны быть выведены с~тремя знаками после запятой. При
этом заголовки должны быть отцентрированы относительно значений 
в~соответствующей колонке.

4.\ Даны координаты трёх вершин треугольника: $ (4, 3) $, $ (7, 7) $, $ (9,
4.5) $. Нужно вычислить его пери\-метр и~площадь. Если непонятно, как решать,
прочитай про формулу Герона. Вывес\-ти вы\-чис\-лен\-ные периметр и~площадь
с~точностью до 4~знаков после запятой.

5.\ Вычислить выражение:
\[
    \sin{\sqrt{x + 1}} - \sin{\sqrt{x - 1}}
\]
при $ x = \pi $ с~точностью 6~знаков после запятой.

6.\ Вычислить выражение:
\[
    |x^2 - x^3| - \frac{7x}{x^3-15x}
\]
при $ x = 4 $.
\end{document}

