\documentclass[a4paper,12pt]{article}
\usepackage[T2A]{fontenc}
\usepackage[utf8]{inputenc}
\usepackage[english,russian]{babel}
\usepackage[margin=2cm]{geometry}
\usepackage{fancyvrb}
\usepackage{inconsolata}
\usepackage{graphicx}

\newcommand\userinput[1]{\textbf{#1}}

\newcounter{problemnumber}

\newenvironment{problem}[1]
	{\addtocounter{problemnumber}{1}\arabic{problemnumber}. (#1)}
	{\vspace{6pt}}

\title{Домашняя работа 29}
\date{}
\begin{document}
\maketitle{}

\emph{Во всех программах рекомендую выделять логику обработки данных в
    функции. Можешь использовать глобальные массивы.}

\begin{problem}{Культин153}
    Написать программу, которая вводит с клавиатуры одномерный массив из 5
    целых чисел и выводит количество ненулевых элементов. Перед вводом каждого
    элемента на экране должна появляться подсказка с его номером.

    \selectlanguage{english}
    \begin{Verbatim}[commandchars=\\\{\}]
    Program counts the number of non-zero elements in the array of 5 integers.
    After entering number press Enter.
    a[1] -> \userinput{12}
    a[2] -> \userinput{0}
    a[3] -> \userinput{3}
    a[4] -> \userinput{-1}
    a[5] -> \userinput{0}

    There are 3 non-zero elements in the array.
    \end{Verbatim}
    \selectlanguage{russian}
\end{problem}

\begin{problem}{Культин154}
    Написать программу, которая выводит минимальный элемент введённого с
    клавиатуры массива целых чисел.
\end{problem}

\begin{problem}{Культин157}
    Написать программу, которая проверяет, находится ли в массиве введённое с
    клавиатуры число. Массив должен вводится во время работы программы, затем
    программа должна запрашивать искомое число.
\end{problem}

\begin{problem}{Культин158}
    Написать программу, которая проверяет, представляют ли элементы
    введённого массива возрастающую последовательность.
\end{problem}

\begin{problem}{Культин159}
    Написать программу, которая проверяет, представляют ли элементы
    введённого массива убывающую последовательность.
\end{problem}

\begin{problem}{Культин160}
    Написать программу, которая проверяет, сколько раз заданное число
    встречается в массиве целых чисел.
\end{problem}

\end{document}
