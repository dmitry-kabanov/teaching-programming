\documentclass[a4paper,12pt]{article}
\usepackage[T2A]{fontenc}
\usepackage[utf8]{inputenc}
\usepackage[english,russian]{babel}
\usepackage[margin=2cm]{geometry}
\usepackage{fancyvrb}
\usepackage{inconsolata}
\usepackage{graphicx}

\newcommand\userinput[1]{\textbf{#1}}

\newcounter{problemnumber}

\newenvironment{problem}[1]
	{\addtocounter{problemnumber}{1}\arabic{problemnumber}. (#1)}
	{\vspace{6pt}}

\title{Домашняя работа 31}
\date{}
\begin{document}
\maketitle{}

\emph{Во всех программах рекомендую выделять логику обработки данных в
    функции. Можешь использовать глобальные массивы.}

\begin{problem}{Семакин7.A.15}
    Дана последовательность чисел, среди которых имеется один нуль. Вывести на
    печать все числа до нуля включительно.
\end{problem}

\begin{problem}{Семакин7.A.19}
    Дан целочисленный массив с количеством элементов $n$ (это число вводится
    пользователем, а сам массив генерируется с помощью генератора случайных
    чисел).  Вывести на печать те элементы массива, индексы которых являются
    степенями двойки: 1, 2, 4, 8$\dots$
\end{problem}

\begin{problem}{Семакин7.A.21}
    Дана последовательность из $N$ вещественных чисел. Вычислить
    \[
        \sigma= \sqrt{\frac{1}{N} \sum_{i=1}^{N} \left(x_i - M \right)^2},
    \]
    где $M$~--- среднее арифметическое всех элементов массива, $x_i$~--- $i$-ый
    элемент массива. Величина $\sigma$ называется среднеквадратическим
    отклонением.  Среднеквадратическое отклонение показывает, насколько
    разбросаны значения в последовательности чисел вокруг среднего значения.
    Чем выше величина среднеквадратического отклонения, тем больше разброс.
\end{problem}

\begin{problem}{Семакин7.A.28}
    Заполнить массив из $N$ элементов с начальным заданным значением A[0]
    $\neq$ 0 по принципу: A[i] = A[i div 2] + A[i-1].
\end{problem}

\end{document}
