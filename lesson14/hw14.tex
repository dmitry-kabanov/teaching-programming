\documentclass[a4paper,12pt,draft]{article}
\usepackage[T2A]{fontenc}
\usepackage[utf8]{inputenc}
\usepackage[english,russian]{babel}
\usepackage[margin=2cm]{geometry}
\usepackage{fancyvrb}
\usepackage{inconsolata}

\newcommand\userinput[1]{\textbf{#1}}

\newcounter{problemnumber}

\newenvironment{problem}[1]
	{\addtocounter{problemnumber}{1}\arabic{problemnumber}. (#1)}
	{\vspace{6pt}}

\title{Домашняя работа 14}
\date{}
\begin{document}
\maketitle{}

\begin{problem}{Семакин4.2.15}
    Дано натуральное число $n$. Вычислить
    \[
        P = \left(1 - \frac{1}{2} \right) \left( 1 - \frac{1}{4} \right)
            \left(1 - \frac{1}{6} \right) \times \dots \times 
            \left( 1 - \frac{1}{2n} \right).
    \]
\end{problem}

\begin{problem}{Семакин4.2.16}
    Дано натуральное число $n > 1$. Вычислить
    \[
        S = 1! + 2! + 3! + \dots + n!.
    \]
\end{problem}

\begin{problem}{Семакин4.2.19}
    Числа Фибоначчи определяются формулами $f_0 = 1$, $f_1 = 1$, $f_k = f_{k-1}
    + f_{k-2}$ при $k = 2, 3, \dots$. Найти $f_p$ для заданного $p$.
\end{problem}

\begin{problem}{Семакин4.5.2м}
    Найти минимальный элемент последовательности. Пользователь должен вводить
    число элементов последовательности, элементы же должны генерироваться с
    помощью генератора случайных чисел. Программа должна выводить саму
    последовательность чисел, а затем минимальный элемент.
\end{problem}

\end{document}
