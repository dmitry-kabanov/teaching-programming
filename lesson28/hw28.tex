\documentclass[a4paper,12pt]{article}
\usepackage[T2A]{fontenc}
\usepackage[utf8]{inputenc}
\usepackage[english,russian]{babel}
\usepackage[margin=2cm]{geometry}
\usepackage{fancyvrb}
\usepackage{inconsolata}
\usepackage{graphicx}

\newcommand\userinput[1]{\textbf{#1}}

\newcounter{problemnumber}

\newenvironment{problem}[1]
	{\addtocounter{problemnumber}{1}\arabic{problemnumber}. (#1)}
	{\vspace{6pt}}

\title{Домашняя работа 28}
\date{}
\begin{document}
\maketitle{}

\emph{Во всех программах требуется написание собственных функций. Функции
    используются для разбиения задачи на подзадачи.}

\begin{problem}{моя}
    Составить программу, которая будет находить простые числа, не превышающие
    заданное число $n$. Организация программы следующая: главная программа
    должна принимать ввод от пользователя и затем вызывать функцию, которая
    будет находить простые числа. Эта функция в свою очередь должна содержать
    только цикл, а проверку на то, является ли число простым, она должна
    поручать ещё одной функции. Таким образом, программа целиком должна
    состоять из главной программы и двух функций.
\end{problem}

\begin{problem}{Семакин, 6.1.B.3}
    Заменить заданное натуральное число числом, получаемым из исходного
    записью его цифр в обратном порядке. Например, задано число 156, требуется
    получить 651.
\end{problem}

\begin{problem}{Семакин, 6.1.B.9}
    Для заданного числа $n$ вычислить сумму
    \[
        1 + \frac12 + \frac13 + \dots + \frac{1}{n},    
    \]
    и представить результат в виде несократимой дроби $\frac{p}{q}$ (где $p$,
    $q$ -- натуральные числа).
\end{problem}

\begin{problem}{Семакин, 6.1.B.18}
    Имеется часть катушки с автобусными билетами. Номер билета шестизначный.
    Определить количество счастливых билетов на катушке, если меньший номер
    билета в катушке $N$, а больший $M$ (оба числа задаются пользователем
    программы). Билет является счастливым, если сумма трёх первых его цифр
    равна сумме трёх последних цифр.
\end{problem}

\end{document}
