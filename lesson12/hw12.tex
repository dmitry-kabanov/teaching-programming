\documentclass[a4paper,12pt,draft]{article}
\usepackage[T2A]{fontenc}
\usepackage[utf8]{inputenc}
\usepackage[english,russian]{babel}
\usepackage[margin=2cm]{geometry}
\usepackage{fancyvrb}
\usepackage{inconsolata}

\newcommand\userinput[1]{\textbf{#1}}

\newcounter{problemnumber}

\newenvironment{zadacha}[1]
	{\addtocounter{problemnumber}{1}\arabic{problemnumber}. (#1)}
	{\vspace{6pt}}

\title{Домашняя работа 12 <<Цикл for ещё раз>>}
\date{}
\begin{document}
\maketitle{}

\emph{Не забудь создать папку hw12 в папке dev!}

\begin{zadacha}{Семакин4.1.12}
Создать числа Пифагора $a, b, c (c^2 = a^2 + b^2)$ на
основе формул $a = m^2 - n^2$, $b = 2mn$, $c = m^2 + n^2$ (где $m, n$)~---
натуральные числа; $1 < m < k$; $1 < n < k$; $k$~--- заданное число). Результат
вывести на экран в виде таблицы из пяти столбцов: $m, n, a, b, c$.
\end{zadacha}

\begin{zadacha}{1}
content...
\end{zadacha}

2. (Семакин4.1.2) Начав тренировки, спортсмен в первый день пробежал 10 км.
Каждый день он увеличивал дневную норму на 10 \% от нормы предыдущего дня.
Определить, какой суммарный путь пробежит спортсмен за $n$ дней. Число $n$
вводится с клавиатуры. Программа должна выводить суммарный путь за каждый
день, от первого до $n$-ого.

3. (Семакин4.1.3) Одноклеточная амёба каждые три часа делится на две клетки.
Определить, сколько амёб будет через 3, 6, 9, 12, \dots, $3n$ часов. Число $n$
вводится с клавиатуры.

4. (Семакин4.1.5) У гусей и кроликов вместе 64 лапы. Сколько может быть
кроликов и сколько гусей (указать все возможные сочетания).

5. (Семакин4.1.17) Найти сумму всех $n$-значных чисел ($1 \leq n \leq 4$).

6. (Семакин4.1.24) Вычислить количество точек с целочисленными координатами,
находящихся в круге с радиусом $R \geq 0 $. Радиус является вещественным
числом, которое пользователь вводит с клавиатуры.

\end{document}
