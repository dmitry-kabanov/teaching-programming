\documentclass[a4paper,12pt,draft]{article}
\usepackage[T2A]{fontenc}
\usepackage[utf8]{inputenc}
\usepackage[english,russian]{babel}
\usepackage[margin=2cm]{geometry}
\usepackage{fancyvrb}
\usepackage{inconsolata}

\newcommand\userinput[1]{\textbf{#1}}

\newcounter{problemnumber}

\newenvironment{problem}[1]
	{\addtocounter{problemnumber}{1}\arabic{problemnumber}. (#1)}
	{\vspace{6pt}}

\title{Домашняя работа 12 <<Цикл for ещё раз>>}
\date{}
\begin{document}
\maketitle{}

\emph{Не забудь создать папку hw12 в папке dev!}

\begin{problem}{Семакин4.1.12}
Создать числа Пифагора $a, b, c (c^2 = a^2 + b^2)$ на
основе формул $a = m^2 - n^2$, $b = 2mn$, $c = m^2 + n^2$ (где $m, n$)~---
натуральные числа; $1 < m < k$; $1 < n < k$; $k$~--- заданное число). Результат
вывести на экран в виде таблицы из пяти столбцов: $m, n, a, b, c$.
\end{problem}

\begin{problem}{Семакин4.2.1}
Дано натуральное число N. Вычислить
\[
    S = 1 + \frac12 + \frac14 - \frac18 + \dots + (-1)^n \cdot \frac{1}{2^n}.    
\]
\end{problem}

\begin{problem}{Семакин4.1.22}
Найти наибольшее значение отношения трёхзначного числа к сумме его цифр.
\end{problem}

\begin{problem}{Семакин4.1.25}
Напечатать в возрастающем порядке все трёхзначные числа, в десятичной записи
которых нет одинаковых цифр, не используя операции деления и нахождения 
остатка от деления.
\end{problem}

\end{document}
