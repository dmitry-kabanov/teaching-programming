\documentclass[12pt,russian,draft]{article}
\usepackage[utf8x]{inputenc}
\usepackage[T1,T2A]{fontenc}
\usepackage[margin=2cm]{geometry}
\usepackage[final]{listings}
\usepackage{fancyvrb}
\usepackage[russian,english]{babel}

\title{Домашняя работа 9 <<Цикл for>>}
\date{}
\begin{document}
\maketitle{}

\emph{Не забудь создать папку hw9 в папке dev!}

1. Разобраться, как работает программа определения, является ли число простым.

2. (Культин102) Написать программу, которая вычисляет сумму первых $n$
натуральных чисел. Количество суммируемых чисел должно вводиться во время
работы программы:
\begin{Verbatim}
Enter n: 9
Sum of first 9 natural numbers equals 45.
\end{Verbatim}

3. (Культин105) Написать программу, которая вычисляет сумму первых $n$ членов
ряда $1 + \frac{1}{2} + \frac{1}{3} + \dots + \frac{1}{n}$. Количество членов
ряда $n$ задаётся пользователем. 

4. (Культин106) Написать программу, которая выводит таблицу степеней двойки (от
нулевой до десятой). Ниже представлен рекомендуемый вывод:
\begin{Verbatim}
 0       1
 1       2
 2       4
...    ...
 9     512
10    1024
\end{Verbatim}

5. (Культин107) Написать программу, которая вычисляет факториал числа $n$,
введённого с клавиатуры. (Факториалом числа $n$ называется произведение целых
чисел 1 до $n$.)

6. (Культин108) Написать программу, которая выводит таблицу значений функции $y
= -2.4x^2 + 5x - 3$ в~диапазоне от -2 до 2 с~шагом 0.5. Ниже представлен
рекомендуемый вид экрана во~время работы программы:
\begin{Verbatim}
   x       y
-----------------
  -2     -22.60
-1.5     -15.90
  -1     -10.40
-0.5      -6.10
   0      -3.00
\end{Verbatim}
и так далее до 2. (\emph{Подсказка: так как в цикле for приращение всегда
    происходит на единицу, то нужно придумать, как обойти здесь это
    ограничение}).

\end{document}
