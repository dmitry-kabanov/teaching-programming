\documentclass[a4paper,12pt,draft]{article}
\usepackage[T2A]{fontenc}
\usepackage[utf8]{inputenc}
\usepackage[english,russian]{babel}
\usepackage[margin=2cm]{geometry}
\usepackage{fancyvrb}
\usepackage{inconsolata}

\newcommand\userinput[1]{\textbf{#1}}

\newcounter{problemnumber}

\newenvironment{problem}[1]
	{\addtocounter{problemnumber}{1}\arabic{problemnumber}. (#1)}
	{\vspace{6pt}}

\title{Домашняя работа 15}
\date{}
\begin{document}
\maketitle{}

\begin{problem}{Семакин4.5.20}
    В последовательности натуральных чисел подсчитать количество чисел,
    оканчивающихся заданной цифрой. Последовательность генерировать с помощью
    генератора случайных чисел.
\end{problem}

\begin{problem}{Семакин4.5.29}
    Определить, между какими степенями двойки расположены элементы
    последовательности. Последовательность генерировать с помощью генератора
    случайных чисел. Для каждого числа нужно выводить степени двойки, между
    которыми это число расположено.
\end{problem}

\begin{problem}{Семакин5.А.9}
    Найти наибольшую и наименьшую цифры в записи заданного натурального числа.
\end{problem}

\begin{problem}{Семакин5.А.15}
    Дано натуральное число $N$. Найти и вывести все числа в интервале от 1 до
    $N$, у которых сумма цифр совпадает с суммой цифр заданного числа. Если
    таких чисел нет, то вывести слово <<нет>>. Например, при $N=44$
    выводятся числа 17, 26 и~35.
\end{problem}

\end{document}
