\documentclass[12pt,russian,draft]{article}
\usepackage[utf8x]{inputenc}
\usepackage[T1,T2A]{fontenc}
\usepackage[margin=2cm]{geometry}
\usepackage[final]{listings}
\usepackage{fancyvrb}

\title{Домашняя работа 5}
\date{}
\begin{document}
\maketitle{}

\emph{В задачах 1--6 необходимо делать следующее. Пользователь вводит
    соответствующие данные. Затем программа проверяет условие, указанное
    в~тексте задачи, и выводит на экран True, если условие истинно, или
    False, если условие ложно.}

1. (Семакин1.3.1) Сумма двух первых цифр заданного четырёхзначного числа равна
сумме двух его последних цифр.

2. (Семакин1.3.5) Введённое целое число является чётным двухзначным числом.

3. (Семакин1.3.6) Треугольник со сторонами $a, b, c$ является равносторонним.

4. (Семакин1.3.7) Треугольник со сторонами $a, b, c$ является равнобедренным.

5. (Семакин1.3.12) Все цифры заданного четырёхзначного числа различны.

6. (Семакин1.3.22) Заданное четырёхзначное число читается одинаково слева
направо и~справа налево.

7. (Культин76) Написать программу, которая вычисляет частное от деления двух чисел. Программа 
должна проверять правильность введённых пользователем данных и, если они неверные (делитель равен нулю), 
выдавать сообщение об ошибке. Примерный ход работы программы:

\begin{Verbatim}
Enter dividend: <пользователь вводит делимое>
Enter divisor: <пользователь вводит делитель>
dividend / divisor is <dividend / divisor>.
\end{Verbatim}

\end{document}
